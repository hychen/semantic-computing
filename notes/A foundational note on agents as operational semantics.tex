\documentclass[11pt]{article}
\usepackage{amsmath, amssymb, amsthm}
\usepackage{hyperref}
\usepackage{geometry}
\geometry{margin=1in}

\title{\textbf{A Foundational Note on Agents as Operational Semantics}}
\author{Hsin-Yi Chen \\ SlashLife AI}
\date{}

\begin{document}
\maketitle

\begin{abstract}
This note presents a minimal formulation in which an agent is the fixed point of a semantic operational law acting on the task structure and its graft. Task structure specifies the abstract organization of activity; graft provides the pragmatic linguistic action required for situated use. Identity, lifecycle, narrative continuity, and attestation arise as higher constructors of this fixed point. The formulation is substrate-agnostic and independent of any particular computational or representational architecture.
\end{abstract}

\section{Premise}

Classical approaches treat an agent as a predefined entity endowed with cognitive or computational attributes. In systems mediated by natural language and contextual action, such an approach obscures the generative process from which agency arises.

A more economical formulation is possible: agency emerges as the homotopy-stable semantic continuation generated by the interaction of abstract task structure with contextual graft.

\section{Task Structure and Graft}

Let \(T\) denote the \emph{task structure}---the abstract, decomposable organization of a goal, workflow, or obligation. Sensory updating requires no separate treatment: admissible perceptual signals appear as morphisms into \(T\).

Let \(G\) denote the \emph{graft}---the pragmatic action that binds a task to its situated context. Graft encompasses deixis, presupposition, discourse role, and other operations ordinarily realized through natural language. It is modality-agnostic: textual, visual, auditory, or behavioural realizations are all treated as semantic representations inhabiting different image spaces of the same core.

Both \(T\) and \(G\) concern semantic form rather than surface expression.

\section{Agents as Fixed Points of a Semantic Operational Law}

Here ``operational semantics'' is used in the semantic rather than computational sense: a law-like operator on semantic states, independent of evaluation strategy or machine model.

I adopt the following definition:
\begin{equation}
\label{eq:agent}
\mathrm{Agent} := \mu X.\,\mathsf{OpSem}(T \times G \to X).
\end{equation}

Equation~\eqref{eq:agent} asserts:

\begin{itemize}
    \item each task--graft pair induces a semantic transformation on a state \(X\);
    \item the agent is the minimal fixed point of this transformation;
    \item agency is a self-stabilizing semantic continuation rather than an ontological object.
\end{itemize}

No commitment is made to Turing machines, symbolic architectures, neural inference systems, or hybrid models; the definition is representation- and substrate-independent.

\section{Higher-Inductive Structure}

The fixed-point formulation supports derived higher structure.

\paragraph{Identity.}
Identity corresponds to a path in the fixed point:
\[
a = a' \;:\equiv\; \mathrm{path\ in}\ \mu X.\,\mathsf{OpSem}(T\times G\to X).
\]

\paragraph{Lifecycle.}
Lifecycle is the inductive unfolding:
\[
a_0 \xrightarrow{(T_1,G_1)} a_1 \xrightarrow{(T_2,G_2)} a_2 \longrightarrow \cdots.
\]

\paragraph{Narrative continuity.}
A higher constructor collects admissible unfoldings into coherent traces.

\paragraph{Attestation.}
Attestation is a dependent type over histories:
\[
\mathsf{Attest}(a) := \Sigma(h : \mathsf{Hist}(a)).\, \mathsf{Verify}(h),
\]
classifying verifiable histories of an agent.

None of these require primitive declaration; they arise from the fixed point itself.

\section{Properties}

Three immediate properties follow:

\begin{enumerate}
    \item \textbf{Minimality.}  
    The formulation relies solely on the fixed-point equation~\eqref{eq:agent}.
    \item \textbf{Universality.}  
    Any agent architecture becomes an instance once its semantic operational law is specified.
    \item \textbf{Pragmatic adequacy.}  
    Graft embeds natural-language pragmatics directly into the semantic process.
\end{enumerate}

\section{Conclusion}

By treating an agent as the fixed point of a semantic operational law acting on task structure and graft, agency is recast as generated rather than presupposed. Identity, lifecycle, narrative structure, and attestation arise without additional primitives. The formulation offers a minimal foundation capable of unifying linguistic pragmatics and computational perspectives on agency while remaining independent of implementation detail.

\end{document}
