\documentclass[11pt]{article}

\usepackage{amsmath,amssymb,amsthm}
\usepackage{hyperref}
\usepackage{enumitem}
\usepackage{geometry}
\usepackage{mathrsfs}
\usepackage{upgreek}
\newtheorem{proposition}{Proposition}
\newtheorem{definition}{Definition}
\geometry{margin=1in}

\title{\textbf{Agents as Undetermined Fixed-Point Combinators:\\
A HoTT-Theoretic Note}}
\author{Hsin-Yi Chen\\SlashLife AI}
\date{}

\begin{document}
\maketitle

\begin{abstract}
This note presents a universe-level formalization of agents as
undetermined fixed-point combinators.
The model is developed within the setting of Homotopy Type Theory (HoTT),
treating agents not as predefined computational objects but
as semantic structures determined by their task composition,
pragmatic grafting, recursive state, and an externally exposed universe endofunctor.
The resulting definition provides a minimal and purely structural foundation
for agents independent of specific representational or computational paradigms.
\end{abstract}

\section{Introduction}

The aim of this note is to provide a minimal, structural
and universe-level formulation of agents grounded in the semantics of
Homotopy Type Theory (HoTT).
Rather than assuming a fixed computational model,
the construction isolates four components:

\begin{enumerate}[label=(\alph*)]
    \item a \emph{task structure} $T$,
    \item a \emph{graft structure} $G$ expressing contextual and pragmatic attachment,
    \item a \emph{recursive agent state} $X$, and
    \item an \emph{undetermined universe endofunctor} $\nabla$.
\end{enumerate}

The essence of the proposal is that an agent is the minimal fixed point
arising from the interaction of these four components.
The formulation is independent of specific representations,
logical formalisms or computational mechanisms.
It is intended as an abstract semantic scaffold that can host
heterogeneous instantiations, including linguistic, social,
procedural or physical variants of agency.

\section{Preliminaries}

We assume a univalent universe $\mathcal{U}$ of types.
For any $u \in \mathcal{U}$ we write $T(u)$ for a task-structural type
depending on $u$, and $G(u)$ for a graft-structural type depending on $u$.

\begin{definition}[Universe Endofunctor]
An \emph{undetermined universe endofunctor} is a map
\[
\nabla : \mathcal{U} \to \mathcal{U},
\]
with no assumptions on internal structure other than closure in $\mathcal{U}$.
\end{definition}

Intuitively, $\nabla(u)$ represents an externally exposed semantic expansion
of $u$, without prescribing how such expansion is generated.
This makes the construction fully agnostic with respect to 
representational, linguistic or physical interpretations.

\section{Combinatorial Construction}

We introduce a four-argument combinator
\[
C : \mathcal{U}^4 \to \mathcal{U},
\]
where
\[
C(T, G, X, Y)
\]
combines task structure, graft structure, recursive state 
and an external universe component.

\begin{definition}[Agent as Undetermined Fixed Point]
\label{def:agent}
An \emph{agent} is defined as the dependent fixed point
\[
\mathrm{Agent} 
:= 
\mu X:\mathcal{U}.\;
C\big(T(X),\, G(X),\, X,\, \nabla(X)\big).
\]
\end{definition}

This definition is universe-level:
$X$ ranges over $\mathcal{U}$, and the construction
selects the smallest solution of the dependent equation.
The fixed point is “undetermined’’ in the sense that
$\nabla$ is arbitrary, and may represent any external trigger,
semantic perturbation or universe extension.

\subsection{Interpretation}

The definition yields four interacting layers:

\begin{itemize}
    \item $T(X)$ encodes obligations, decompositions and procedural structure
          dependent on the agent's own state.
    \item $G(X)$ provides pragmatic anchoring, capturing contextual,
          cultural or linguistic grafts on $X$.
    \item $X$ contributes the recursive self-modeling needed for persistence.
    \item $\nabla(X)$ denotes an external universe-level perturbation
          affecting $X$, with no constraints on modality or mechanism.
\end{itemize}

Thus, the fixed point expresses the minimal semantic object
capable of sustaining the interaction of these components.

\section{Properties}

\begin{proposition}[Universality]
Any structure admitting task, graft, recursive and external components
can be embedded into an instance of Definition~\ref{def:agent}.
\end{proposition}

\begin{proposition}[Agnosticism]
The construction does not assume a specific logic, representation,
reasoning method, medium or computational substrate.
All such choices are pushed into $T$, $G$ and $\nabla$.
\end{proposition}

\begin{proposition}[Stability]
If $\nabla$ preserves equivalences in $\mathcal{U}$,
then the agent fixed point is stable under univalent identification.
\end{proposition}

These properties reflect the goal of providing a minimal 
HoTT-theoretic basis for semantic agency.

\section{Discussion}

The use of $\nabla$ is essential.
By abstracting external semantic influence to an endofunctor,
the definition avoids premature commitments to 
syntactic, symbolic or physical mechanisms.
Similarly, graft structure $G$ isolates pragmatic dependence:
context, convention and interpretation do not pollute 
the task structure itself.

The model simultaneously admits:

\begin{itemize}
    \item linguistic interpretations,
    \item procedural interpretations,
    \item logical interpretations,
    \item physical or dynamical interpretations.
\end{itemize}

These arise by instantiating $T$, $G$ and $\nabla$ appropriately.

\section{Conclusion}

This note has presented a universe-level formulation of agents
as undetermined fixed-point combinators.
The model is intentionally minimal, abstract
and independent of any specific representational or computational paradigm.
Its purpose is to provide a semantic scaffold for further development,
including possible elaborations involving 
lifecycle operators, narrative structures,
attestation layers or pathway types.

\bibliographystyle{plain}
\begin{thebibliography}{9}

\bibitem{hott}
The Univalent Foundations Program.
\emph{Homotopy Type Theory: Univalent Foundations of Mathematics}, 2013.

\bibitem{awodey}
Awodey, S.
\emph{Category Theory}, Oxford University Press.

\end{thebibliography}

\end{document}
